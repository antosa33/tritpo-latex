\documentclass[aspectratio=169, 12pt]{beamer}

\usepackage{lmodern}
\usepackage[utf8]{inputenc}
\usepackage[english,russian]{babel}
\usepackage{hyperref}

\hypersetup{
    colorlinks=true,
    linkcolor=blue,
    filecolor=blue,
    urlcolor=blue,
}

\setbeamertemplate{navigation symbols}{}

\usecolortheme{lily}
\usetheme{CambridgeUS}

\title{ТРиТПО}
\subtitle{Технологии разработки и тестирования программного обеспечения}
\author{Artsiom Vasilevich}
\institute[BSUIR]{Belarusian State University of Informatics and Radioelectronics}
\date{\tiny \today}

\begin{document}

\frame{\titlepage}

\begin{frame}
    \frametitle{Введение в курс}
    \framesubtitle{123}

\end{frame}

\begin{frame}
    \frametitle{Введение в курс}
    \framesubtitle{Лабораторные занятия}
    \begin{itemize}
        \item 8 общих занятий
        \item 8 занятий в подгруппах
    \end{itemize}
\end{frame}

\begin{frame}
    \frametitle{Введение в курс}
    \framesubtitle{Путь к <<успеху>>}
    \begin{itemize}
        \item 6 лабораторных работ
        \item 1 --- 4 индивидуальные
        \item 5 --- 6 в парах
    \end{itemize}
\end{frame}

\begin{frame}
    \frametitle{Введение в курс}
    \framesubtitle{Система оценок}
    \begin{itemize}
        \item максимальная оценка за лр 1.0 \pause
        \item учёт ведётся с помощью установленных deadline \pause
        \begin{itemize}
            \item дата выдачи - start-date 10.09 \pause
            \item soft deadline для каждой из подгрупп 17.09 \pause
            \item hard deadline = start-date + $\sim$3 недели \pause
            \item лр сдана в soft deadline: $\leq$ 1.0
            \item лр сдана до hard но после soft: 0.2 --- 0.8
            \item лр сдана в hard или позже: $\leq$ 0.2
        \end{itemize}
    \end{itemize}
\end{frame}

\begin{frame}
    \frametitle{Введение в курс}
    \framesubtitle{Коротко о лабораторных} \pause
    \begin{enumerate}
        \item Java, JUnit, Git \pause
        \item Project SRS \pause
        \item Use-case, Activity, State diagrams \pause
        \item Class, Sequence, Component/Deployment diagrams \pause
        \item Project implementation + \textbf{patterns} + code-review \pause
        \item Test-cases and test-plan
    \end{enumerate}
\end{frame}

\begin{frame}[t]
    \frametitle{Введение в курс}
    \framesubtitle{Лабораторная работа \textnumero 1}
    Порядок выполнения (1/2): \pause
    \begin{itemize}
        \item Изучить теоретические сведения \pause
        \item Установить git (если Вы этого ещё не сделали) \pause
        \item Создать аккаунт на GitHub (если его ещё нет) \pause
        \item Fork repository \url{https://github.com/trtpo/laba1} \pause
        \item Clone repository --- git clone url \pause
        \item Собрать проект, запустить, проанализировать полученный результат
    \end{itemize}
\end{frame}

\begin{frame}[t]
    \frametitle{Введение в курс}
    \framesubtitle{Лабораторная работа \textnumero 1}
    Порядок выполнения (2/2): \pause
    \begin{itemize}
        \item Внести изменения в код (каждое изменение – отдельный commit) \pause
        \begin{itemize}
            \item изменить цветовую гамму изображения \pause
            \item изменить уравнение фрактала \pause
            \item расширить набор операций над комплексными числами и использовать их в новом уравнении \pause
            \item добавить Unit тесты для проверки правильности новых операций \pause
        \end{itemize}
        \item Push изменений в origin \pause
        \item Создать pull-request
    \end{itemize}
\end{frame}

\begin{frame}[t]
    \frametitle{Введение в курс}
    \framesubtitle{Лабораторная работа \textnumero 1}
    Вопросы:
    \begin{enumerate}
        \item 1
        \item 2
        \item 3
        \item 4
        \item 5
    \end{enumerate}
\end{frame}

\begin{frame}[t]
    \frametitle{Введение в курс}
    \framesubtitle{Ресурсы}
    \begin{thebibliography}{10}
        \bibitem{ProGit}[Goldbach, 1742]
        Christian Goldbach.
        \newblock A problem we should try to solve before the ISPN ’43 deadline,
        \newblock \emph{Letter to Leonhard Euler}, 1742.
        \bibitem{Java}[Java, 2020]
    \end{thebibliography}
\end{frame}

\end{document}
