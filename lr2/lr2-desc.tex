\documentclass[aspectratio=169, 12pt]{beamer}

\usepackage{lmodern}
\usepackage[utf8]{inputenc}
\usepackage[english,russian]{babel}
\usepackage{hyperref}

\hypersetup{
    colorlinks=true,
    linkcolor=blue,
    filecolor=blue,
    urlcolor=blue,
}

\setbeamertemplate{navigation symbols}{}

\usecolortheme{lily}
\usetheme{CambridgeUS}

\title{ТРиТПО}
\subtitle{Технологии разработки и тестирования программного обеспечения}
\author{Artsiom Vasilevich}
\institute[BSUIR]{Belarusian State University of Informatics and Radioelectronics}
\date{\tiny \today}

\begin{document}

\frame{\titlepage}

\begin{frame}
    \frametitle{Software Requirements}
    \framesubtitle{Понятие}
    IEEE Standard Glossary of Software Engineering Terminology (1990)
    определяет <<программные требования>> как: \pause
    \begin{enumerate}
        \item Условия или возможности, необходимые пользователю для решения
              проблем или достижения целей; \newline \pause
        \item Условия или возможности, которыми должна обладать система или
              системные компоненты, чтобы выполнить контракт или удовлетворять
              стандартам, спецификациям или другим формальным документам; \newline \pause
        \item Документированное представление условий или возможностей для
              пунктов 1 и 2.
    \end{enumerate}
\end{frame}

\begin{frame}
    \frametitle{Software Requirements}
    \framesubtitle{Проблемы на проектах, связанные с требованиями} \pause
    \begin{itemize}
        \item Пропустили заинтересованных лиц (ЗЛ) \pause
        \item Недостаточное вовлечение ЗЛ \pause
        \item Расползание рамок проекта (scope creep) \pause
        \item Некачественные требования \pause
        \item Разработка ненужного функционала (gold plating) \pause
        \item Поспешные оценки и планирование
    \end{itemize}
\end{frame}

\begin{frame}[t]
    \frametitle{Software Requirements}
    \framesubtitle{Виды требований по уровням} \pause
    Бизнес-требования --- определяют назначение ПО, описываются в документе о
    видении (vision) и границах проекта (scope). \newline \pause

    Пользовательские требования --- определяют набор пользовательских задач, которые должна
    решать система, а также способы (сценарии) их решения. Пользовательские требования
    могут выражаться в виде фраз утверждений, в виде вариантов ипользования (use case),
    пользовательских историй (user story), сценариев взаимодействия (scenario). \newline \pause

    Функциональные требования --- определяют <<как>> реализовать продукт. Описываются в системоной
    спецификации (System Requirement Specification, SRS).
\end{frame}

\begin{frame}
    \frametitle{Software Requirements}
    \framesubtitle{Виды требований по характеру} \pause
    Функциональные --- требования к поведению системы, продукта \newline \pause

    Нефункциональные --- требования к характеру поведения системы \pause
    \begin{itemize}
        \item Бизнес-правила
        \item Атрибуты качества
        \item Внешние интерфейсы
        \item Ограничения
    \end{itemize}
\end{frame}

\begin{frame}[t]
    \frametitle{Software Requirements}
    \framesubtitle{Источники требований}
    \begin{itemize}
        \item Заинтересованные стороны (stakeholders) и опытные пользователи (Power Users) \pause
        \item Федеральное и муниципальное отраслевое законодательство
              (конституция, законы, распоряжения) \pause
        \item Нормативное обеспечение организации (регламенты, положения, ...) \pause
        \item Текущая организация деятельности объекта автоматизации \pause
        \item Модели деятельности (диаграммы бизнес-процессов) \pause
        \item Представления и ожидания потребителей и пользователей системы \pause
        \item Журналы пользования существующих программно-аппаратных систем \pause
        \item Конкурирующие программные продукты
    \end{itemize}
\end{frame}

\begin{frame}[t]
    \frametitle{Software Requirements}
    \framesubtitle{Характеристики качественных требований}
    Единичность \newline
    \begin{itemize}
        \item Требования описывает одну и только одну вещь \newline \pause
        \item Мне (как заказчику) важны все возможности интернет-магазина.
              Всё должно быть готово как можно быстрее.
    \end{itemize}
\end{frame}

\begin{frame}[t]
    \frametitle{Software Requirements}
    \framesubtitle{Характеристики качественных требований}
    Завершенность \newline
    \begin{itemize}
        \item Требование полностью определено в одном месте и вся необходимая информация присутствует \newline \pause
        \item Страница <<Каталог>> должна отображать список категорий товаров (телевизоры, компьютеры, ??? TBD),
              предлагаемых интернет-магазином.
    \end{itemize}
\end{frame}

\begin{frame}[t]
    \frametitle{Software Requirements}
    \framesubtitle{Характеристики качественных требований}
    Последовательность \newline
    \begin{itemize}
        \item Требование не противоречит другим требованиям и полностью соответствует внешней документации \newline \pause
    \end{itemize}
    Атомарность \newline
    \begin{itemize}
        \item Требование не может быть разбито на ряд более детальных требования без потери завершенности
    \end{itemize}
\end{frame}

\begin{frame}[t]
    \frametitle{Software Requirements}
    \framesubtitle{Характеристики качественных требований}
    Отслеживаемость \newline
    \begin{itemize}
        \item Требование полностью или частично соответсвует деловым нуждам как заявлено ЗЛ и задокументировано \newline \pause
    \end{itemize}
    Актуальность \newline
    \begin{itemize}
        \item Требование не стало устаревшим с течением времени
    \end{itemize}
\end{frame}

\begin{frame}[t]
    \frametitle{Software Requirements}
    \framesubtitle{Характеристики качественных требований}
    Выполнимость \newline
    \begin{itemize}
        \item Требование может быть реализовано в пределах проекта \newline \pause
        \item В интернет-магазине должна быть реализована возможность запуска Internet Explorer на стороне
              пользователя, в случае если он зашёл на сайт под другим браузером.
    \end{itemize}
\end{frame}

\begin{frame}[t]
    \frametitle{Software Requirements}
    \framesubtitle{Характеристики качественных требований}
    Недвусмысленность \newline \pause
    \begin{itemize}
        \item Требование кратко определено без обращения к техническому жаргону, акронимам
              и другим скрытым формулировкам \pause
        \item Выражает объективные факты, не субъективные мнения (<<удобный интерфейс>>);
              возможна одна и только одна интерпретация \pause
        \item Определение не содержит нечётких фраз \pause
        \item Не содержит отрицательных утверждений и составных утверждений
    \end{itemize}
\end{frame}

\begin{frame}[t]
    \frametitle{Software Requirements}
    \framesubtitle{Характеристики качественных требований}
    Обязательность \newline
    \begin{itemize}
        \item Требование представляет определённую заинтересованным лицом характеристику,
              отсутствие которой приведёт к неполноценности решения, которая не может быть проигнорирована \newline \pause
        \item ! необязательное требование --- противоречие самому понятию требования !
    \end{itemize}
\end{frame}

\begin{frame}[t]
    \frametitle{Software Requirements}
    \framesubtitle{Характеристики качественных требований}
    Проверяемость \newline
    \begin{itemize}
        \item Реализованность требования может быть определена через один из четырёх возможных методов:
              осмотр, демонстрация, тест или анализ \newline \pause
        \item Система должна обладать хорошей производительностью и быстро загружаться в браузере.
    \end{itemize}
\end{frame}

\begin{frame}[t]
    \frametitle{Лабораторная работа \textnumero 2}
    \framesubtitle{Порядок выполнения (1/2)} \pause
\end{frame}

\begin{frame}[t]
    \frametitle{Лабораторная работа \textnumero 2}
    \framesubtitle{Порядок выполнения (2/2)}
\end{frame}

\begin{frame}[t]
    \frametitle{Лабораторная работа \textnumero 2}
    \framesubtitle{Вопросы} \pause
\end{frame}

\begin{frame}[t]
    \frametitle{Лабораторная работа \textnumero 2}
    \framesubtitle{Ресурсы}
    \begin{thebibliography}{10}
        \beamertemplatebookbibitems
        \bibitem{ProGit}
        Pro Git, Second Edition.
        \newblock {\em Open source comprehesive git book}.
        \newblock Scott Chacon, Ben Straub, Community, 2020.
        \bibitem{Java}
        Java: The Complete Reference, Eleventh Edition.
        \newblock {\em Book explains how to develop, compile, debug, and run Java programs}.
        \newblock Herbert Schildt, 2018.
        \bibitem{Lections}
        ЭУМКД ТРиТПО.
        \newblock {\em Лекционный материал по дисциплине ТРиТПО}.
        \newblock Наталья Искра, 2019.
    \end{thebibliography}
\end{frame}

\end{document}