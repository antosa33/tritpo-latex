\documentclass[12pt, a4paper]{article}

\usepackage[utf8]{inputenc}
\usepackage[english,russian]{babel}
\usepackage{setspace}
\usepackage{geometry}
    \geometry{
    a4paper,
    total={170mm,257mm},
    left=15mm,
    top=20mm,
}

\usepackage{hyperref}
\hypersetup{
    colorlinks=true,
    linkcolor=blue,
    filecolor=magenta,
    urlcolor=blue,
}

\begin{document}

\singlespacing

\title{Лабораторная работа \textnumero 2}
\date{\normalsize \today}
\maketitle

\noindent
<<Разработка и документирование требований к учебному проекту>> \newline

\noindent
Цели:
\begin{itemize} \itemsep0em
    \item изучить процесс выявления требований
    \item научиться документировать требования
    \item получить навыки совместной работы с требованиями
\end{itemize}

\noindent
Рабочие языки: русский или английский  \newline

\noindent
Инструментальные средства:
\begin{itemize} \itemsep0em
    \item для документирования требований: текстовый редактор / Visual Paradigm (CRC, brainstorming ...)
    \item для прототипирований GUI: графический редактор / Balsamiq Mockup
    \item GitHub
\end{itemize}

\noindent
Порядок выполнения работы (работа в парах, задание индивидуальное!):
\begin{enumerate} \itemsep0em
    \item Изучить теоретический материал (прослушать лекцию, ознакомиться с литературой)
    \item Внимательно прочитать задание
    \item Определить тип проекта (обсудить с напарником, а затем согласовать с преподавателем) - Web / Desktop / Mobile - приложение,
          использование внешних сервисов, предпочтительные технологии, границы, язык описания и язык разработки;
          придумать название
    \item Заполнить шаблон документа (на русском или английском языке)
    \item Создать репозиторий на Github и залить туда документы - SRS и мокапы, в PDF или Markdown!!!
    \item Ознакомиться с соответствующим заданием напарника, отписать
          комментарии по теме (замечания, предложения - конструктивные)
    \item Прислать ссылку на свой репозиторий на адрес \href{mailto:avasilevich.work@gmail.com}{avasilevich.work@gmail.com}
          (от реального имени и с номером группы)
    \item Держать репозиторий up-to-date!
\end{enumerate}

\noindent
Бонусы тем, кто поучаствует в обсуждении более чем одного, кроме своего, проекта (конструктивно) \newline

\newpage
\noindent
Варианты заданий (лучше собственные):

\begin{enumerate} \itemsep0em
    \item Приложение для предоставления информации о курсе различных
          валют / ценных бумаг / золотовалютных резервов (использовать
          актуальные данные с доступных сервисов в Интернет). В первую
          очередь, выявить бизнес-цели (кому и зачем такое приложение
          может понадобиться). Продумать, какие валюты показывать (или
          предоставить пользователю право выбирать самому – исходя из
          бизнес-целей), нужно ли показывать какую-либо дополнительную
          информацию – архивные данные, графики и т.д.
          \begin{itemize} \itemsep0em
              \item архивные курсы для расчёта налогообложения индивидуального
                    предпринимателя
              \item средство анализа финансовых данных (визуализация колебаний
                    курсов, выявление корреляции между различными финансовыми
                    данными)
              \item расчёт выгодности вкладов (с учётом исторических данных)
          \end{itemize}
    \item Приложение для предоставления информации, связанной с
          местоположением (можно использовать открытые API типа
          forsquare, googlemaps, yandex и т.п. – изучить возможности и
          обосновать выбор, или исходя из желания разработчика). В первую
          очередь, выявить бизнес-цели (кому и зачем такое приложение
          может понадобиться).
          \begin{itemize} \itemsep0em
              \item прокладка маршрута по интересным местам для проведения
                    экскурсии (можно использовать дополнительную информацию со
                    справочных ресурсов)
              \item планирование дня, исходя из списка важных дел (поиск ближайшего
                    магазина / аптеки / СТО и т.д.)
          \end{itemize}
    \item Приложение для предоставления информации о просмотренных
          фильмах, книгах, купленных товарах (можно использовать данные
          сервисов imdb, amazon, goodreads)
          \begin{itemize} \itemsep0em
              \item менеджер личной коллекции фильмов / книг
              \item сервис рекомендаций фильмов/книг
              \item планирование пополнения личной библиотеки
          \end{itemize}
    \item Список дел/ калькулятор/ список покупок / расписание /
          напоминалка (с «изюминкой», которая определяется бизнес-целями)
    \item Многопользовательские игры (с бизнес-целями)
\end{enumerate}

\end{document}
